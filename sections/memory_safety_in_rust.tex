\section{Memory Safety in Rust}

%%%

\begin{frame}{Ownership}

Rust's central feature is ownership.
Although the feature is straightforward to explain,
it has deep implications for the rest of the language.

All programs have to manage the way they use a
computer's memory while running. Some languages
have garbage collection that constantly looks for
no longer used memory as the program runs;
in other languages, the programmer must explicitly
allocate and free the memory.

Rust uses a third approach: memory is managed
through a system of ownership with a set of rules
that the compiler checks at compile time. None of
the ownership features slow down your program
while it's running.

\end{frame}

%%%

\begin{frame}[label={threerules}]{Three Rules}

\begin{block}{Rule 1}
Each value in Rust has a variable that’s called its owner.
\end{block}
\begin{block}{Rule 2}
There can only be one owner at a time.
\end{block}
\begin{block}{Rule 3}
When the owner goes out of scope, the value will be dropped.
\end{block}

\end{frame}

%%%

\begin{frame}[fragile]{Move, Copy, Clone}

\begin{minted}{rust}
let s1 = String::from("hello");
let s2 = s1;

println!("{}, world!", s1);
\end{minted}

\sep

\begin{minted}[fontsize=\footnotesize]{text}
error: use of moved value: `s1`
let s2 = s1;
  -- value moved here

println!("{}, world!", s1);
  ^^ value used here after move

note: move occurs because `s1` has type `std::string::String`,
which does not implement the `Copy` trait
\end{minted}

\end{frame}

%%%

\begin{frame}{C++ Comparison}
	\begin{itemize}
    \item std::unique\_ptr $\rightarrow$ Rust's Box
    \item std::shared\_ptr $\rightarrow$ Rust's Arc
    \item no std::move in Rust. Moves are the default for non-Copy types
    \item no rvalue references in Rust $\rightarrow$ no std::forward
    \item only similarity between Rust references and C++ references is that they are non-null and always point to a valid object
	\end{itemize}
\end{frame}